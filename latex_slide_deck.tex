\documentclass{beamer}
\usepackage[latin1]{inputenc}
\usepackage{listings}
\usetheme{Warsaw}
\usecolortheme{wolverine}
\title{LaTeX Primer}
\author{Clint Grimsley clint.grimsley@gmail.com}
\date{September 29, 2012}
\begin{document}

\begin{frame}
\titlepage
\end{frame}

\begin{frame}
  \frametitle{Outline}
    \tableofcontents
\end{frame}

\section{LaTeX, isn't that paint?}
\subsection{What is LaTeX}

\begin{frame}
\frametitle{What is LaTeX}
\begin{itemize}
  \item TeX is a typesetting program
  \item LaTeX is a document preparation system
  \item The result of compiling a document created in LaTeX is
    typically a PDF document
  \item it's all plaintext, which means it has awesome features
    \begin{itemize}
    \item version controllable with things like git, mercurial,
      subversion (gross) and CVS (grosser)
      \item compressable
      \item easily ported and emailed
      \item highly programmatic
      \item encryptable with GnuPG
    \end{itemize}
\end{itemize}
\end{frame}

\subsection{More reasons LaTeX is awesome}
\begin{frame}
  \frametitle{More reasons LaTeX is awesome}
  \begin{itemize}
    \item Keeps awesome bibliographies (BibTeX)
    \item Can be made aware of different journal formats
      \begin{itemize}
        \item APA
        \item Chicago Style
        \item MLA
      \end{itemize}
    \item Programmatic
      \begin{itemize}
      \item Means you can use fun languages like Python to generate
        really good looking reports from SQL databases
      \item Or you can use with other usual suspects, like emacs
        org-mode and DocBook
      \end{itemize}
  \end{itemize}
\end{frame}

\section{Tell me more!}
\subsection{BibTeX}
\begin{frame}
  \frametitle{Tell me more!}
  \begin{itemize}
    \item BibTex
      \begin{itemize}
        \item Keep bibliographies in an easy-to-read plaintext document
        \item example:
          \lstinputlisting{sample-bibtex.tex}
        \item The beginning (after the @Book) is the reference
          identifier, this can be used to create footnotes in your
          articles that will provide hyperreferential links within the
          resulting PDF document, as well as do things like
          paranthetical citations
        \item the rest is self-explanatory
      \end{itemize}
  \end{itemize}
\end{frame}

\subsection{Tables}
\begin{frame}
  \frametitle{Tables}
  \begin{itemize}
  \item Tables are the cornerstone of good reporting
  \item Be forewarned, tables take practice
  \end{itemize}
    \begin{table}
    \centering
    \begin{tabular}{|l|l|}
      \hline
      \multicolumn{2}{|c|}{Sample Table} \\
      \hline
      Data & 1 \\
      Data & 2 \\
      \hline
    \end{tabular}
  \end{table}
\end{frame}

\begin{frame}
  \frametitle{Table Sample Code}  
  \lstinputlisting{table_sample.tex}  
\end{frame}

\end{document}